\documentclass[a4paper]{article}
\usepackage{titling}
\usepackage{authblk}
\usepackage{fancyhdr}
\usepackage{hyperref}
\usepackage{rsc}
\usepackage{siunitx}
\usepackage{graphicx}
\usepackage{listings}
\usepackage{color}

\definecolor{dkgreen}{rgb}{0,0.6,0}
\definecolor{gray}{rgb}{0.5,0.5,0.5}
\definecolor{mauve}{rgb}{0.58,0,0.82}

\lstset{frame=tb,
  language=Python,
  aboveskip=3mm,
  belowskip=3mm,
  showstringspaces=false,
  columns=flexible,
  basicstyle={\ttfamily},
  numbers=none,
  numberstyle=\tiny\color{gray},
  keywordstyle=\color{blue},
  commentstyle=\color{dkgreen},
  stringstyle=\color{mauve},
  breaklines=true,
  breakatwhitespace=true,
  tabsize=3
}
\DeclareSIUnit\Fahrenheit{\degree F}

\title{Lecture 3: Lists, arrays, and optimisation with NumPy}
\author[1]{Dr Benjamin J. Morgan}
\author[1,2]{Dr Andrew R. McCluskey}
\affil[1]{Department of Chemistry, University of Bath, email: b.j.morgan@bath.ac.uk}
\affil[2]{Diamond Light Source, email: andrew.mccluskey@diamond.ac.uk}
\setcounter{Maxaffil}{0}
\renewcommand\Affilfont{\itshape\small}

\pagestyle{fancy}
\fancyhf{}
\rhead{CH40208}
\lhead{\thetitle}
\rfoot{\thepage}

\begin{document}
\maketitle

\section*{Aim}
This lecture will introduce lists, arrays, and show how the NumPy library can be used to make your code faster.

\section{Lists}
The Python programming language natively includes the ability to group together a series of objects.
These are \texttt{lists} and are one of the most powerful Python objects.
Lists are an ordered set of objects, from which it is possible to pick all, one, or many values.
A list is defined as follows,
\begin{lstlisting}
# Making a list

elements = ["Hydrogen", "Helium", "Lithium", "Beryllium", "Boron", "Carbon", "Nitrogen", "Oxygen"]
\end{lstlisting}
Having defined the list, it is then possible to select individual items of the list by using the following syntax,
\begin{lstlisting}
# Printing some items

print(elements[0], elements[4], elements[-1])
\end{lstlisting}
Note, that Python starts counting from the number $0$, and using the minus sign we can ask Python to count from the end.
This means that the above code should print, \texttt{"Hydrogen", "Boron", "Oxygen"}.
This counting from $0$ means that in the above list, the string \texttt{"Hydrogen"} would be referred to as the zeroth object in the list, while \texttt{"Helium"} would be the first.

In addition to making use of single objects from within a list, it is also possible to create sublists, for example,
\begin{lstlisting}
# Just the first 4 elements

print(elements[0:4])
\end{lstlisting}
Note that above, the numbers on either side of the colon the list indices.
However, rather strangely, the sublist created is \textbf{inclusive} of the first number and \textbf{exclusive} of the second.
Additionally, it is possible to select non-consecutive objects from a list by placing commas between the indices,
\begin{lstlisting}
# Just the gases

print(elements[0, 1, 6, 7])
\end{lstlisting}
The final point about \texttt{lists} is that the data that they hold does not all need to be the same time.
For example, the list below contains a \texttt{float}, two \texttt{str}, a \texttt{complex} number and an \texttt{int},
\begin{lstlisting}
# List of many types

a_new_list = ['hello', 12.41242, 5 + 8j, 'sadness', 2]
print(a_new_list)
\end{lstlisting}

\section{NumPy Arrays}

NumPy (or \texttt{numpy} or more commonly \texttt{np}) is a library that Python can use that is designed and optimised for doing numerical operations.\cite{numpy}
Over this course you will be introduced to many other Python libraries, in order to use any of these you must \texttt{import} them,
\begin{lstlisting}
# Import NumPy

import numpy as np
\end{lstlisting}
This asks the Python interperator to go and find the NumPy library, then in order to reduce the amount of typing (programmers are lazy), we give the library the alias \texttt{np}.

One of the most powerful features of the NumPy library is the \texttt{array}, these are similar to lists but with some important differences.
Unlike a list, all of the items in a NumPy array must be of the same type; namely a NumPy data type (a list of these can be found online: \url{https://docs.scipy.org/doc/numpy/user/basics.types.html}) which are numerical data types such as \texttt{int}, \texttt{float}, and \texttt{complex}.

The power of a NumPy array comes in the ability to perform mathematical operations incredibly efficiently.
For example,\footnote{When running on a MacBook Air 2018 with a 1.6 GHz Intel Core i5.} the summation of zero to ten million is $\sim 25$ times faster when using the NumPy array operation shown below when compared with a simple implementation in pure Python,
\begin{lstlisting}
# The pure Python way

numbers = range(10000000)
total = 0
for i in numbers:
    total = total + i
print(total)

# The NumPy operation

import numpy as np

numbers = np.arange(10000000)
total = np.sum(numbers)
print(total)
\end{lstlisting}
Note that in the above example, the \texttt{range} function creates a list of numbers from $0$ to $10000000$, while the \texttt{np.arange} function creates a NumPy array containing the same values.

NumPy arrays also have a \emph{huge} amount of additional functionality, such as the ability to easily access statistically relevant values, powerful sub-array definition (in particular for multi-dimensional arrays), data reorganisation.
Some of these tools are shown below, and no doubt you will become familiar with many others throughout this course,
\begin{lstlisting}
# Determine the mean and standard deviation
import numpy as np

## First get an array of 6 random ints from 0 to 10
x1 = np.random.randint(10, size=6)

print(x1.mean(), x1.std())

## Now get a two-dimensional array of random ints
x2 = np.random.randint(10, size=(3, 2))

print(x2.shape)
print(x2)
print(x2[0])
print(x2[:, 1])
print(x2[:, 0::-1])

## Lets reshape a one-dimensional array
x3 = np.arange(10)

print(x3)
print(x3.reshape((3, 3))
\end{lstlisting}

\section{Copying lists/arrays}

An important fact to be aware of for both lists and NumPy arrays is that assigning a list to an new variable does \textbf{not} create a new list.
Rather, this will create an alias to the same object in memory.
In order to create a new list (or array), it is best to \texttt{copy} the original object to the new variable, as shown below,
\begin{lstlisting}
# Copying lists and arrays

my_list = ['dog', 'cat', 'horse']

new_list = my_list

new_list[0] = 'giraffe'

print(new_list)
print(my_list)

copied_list = my_list.copy()

copied_list[1] = 'pig'

print(copied_list)
print(my_list)
\end{lstlisting}
Note that for a NumPy array, the function \texttt{np.copy(my\_array)} should be used.


\bibliographystyle{rsc}
\bibliography{handout_3}

\end{document}
