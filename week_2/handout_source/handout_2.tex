\documentclass[a4paper]{article}
\usepackage{titling}
\usepackage{authblk}
\usepackage{fancyhdr}
\usepackage{hyperref}
\usepackage{rsc}
\usepackage{siunitx}
\usepackage{graphicx}
\usepackage{listings}
\usepackage{color}

\definecolor{dkgreen}{rgb}{0,0.6,0}
\definecolor{gray}{rgb}{0.5,0.5,0.5}
\definecolor{mauve}{rgb}{0.58,0,0.82}

\lstset{frame=tb,
  language=Python,
  aboveskip=3mm,
  belowskip=3mm,
  showstringspaces=false,
  columns=flexible,
  basicstyle={\ttfamily},
  numbers=none,
  numberstyle=\tiny\color{gray},
  keywordstyle=\color{blue},
  commentstyle=\color{dkgreen},
  stringstyle=\color{mauve},
  breaklines=true,
  breakatwhitespace=true,
  tabsize=3
}
\DeclareSIUnit\Fahrenheit{\degree F}

\title{Lecture 2: Pythonic logic and loops}
\author[1]{Dr Benjamin J. Morgan}
\author[1,2]{Dr Andrew R. McCluskey}
\affil[1]{Department of Chemistry, University of Bath, email: b.j.morgan@bath.ac.uk}
\affil[2]{Diamond Light Source, email: andrew.mccluskey@diamond.ac.uk}
\setcounter{Maxaffil}{0}
\renewcommand\Affilfont{\itshape\small}

\pagestyle{fancy}
\fancyhf{}
\rhead{CH40208}
\lhead{\thetitle}
\rfoot{\thepage}

\begin{document}
\maketitle

\section*{Aim}
In this lecture, you will learn about logical operations, conditional statements, and for and while loops.

\section{Logical operators}

Previously, we observed how to use Python or a Jupyter Notebook as a simple calculator. 
However, programs become more useful when we are able to make the program more ``intelligent'' allowing it to perform different calculations under different circumstances.
This ability relies heavily on the use of logical operators, as we shall see. 
A logical operator is code that returns a Boolean, either \texttt{True} or \texttt{False}. 
The logical operator \texttt{==} is one of the most common, and translates to \emph{is equal to}, this operator will return \texttt{True} if the values on either side are the same, for example,
\begin{lstlisting}
# The truth

print(14.0/2.0 == 7.0)
print(13.0/2.0 == 7.0)
\end{lstlisting}
This code will return \texttt{True} then \texttt{False}. 

The equals operator is only one of many logical operators that is available in Python, some are given in Table~\ref{tab:ops}.
Each of these may be used in the same syntax as the equals operator.
\begin{table}[h]
	\centering
	\caption{Some logical operators available in Python.}
	\label{tab:ops}
	\begin{tabular}{c c c c}
		\hline
		Name & Mathematical Symbol & Operator & Translation \\
		\hline
		Equals & $=$ &\texttt{==} & \emph{is equal to} \\
		Less than & $<$ & \texttt{<} & \emph{is less than} \\
		Less than or equal & $\leq$ & \texttt{<=} & \emph{is less than or equal to} \\
		Greater than & $>$ & \texttt{>} & \emph{is greater than} \\
		Greater than or equal to & $\geq$ & \texttt{>=} & \emph{is greater than or equal to} \\
		Not equal & $\neq$ & \texttt{!=} & \emph{is not equal to} \\
		\hline
	\end{tabular}
\end{table}
\vspace{\baselineskip}
\begin{center}
	\noindent\fbox{%
		\begin{minipage}{0.9\textwidth}%
			\vspace{0.15\baselineskip}
			\subsubsection{Exercise}
			\begin{itemize}
				\item{Write code that will return the result of the following logical operations:}
				\begin{enumerate}
					\item{$1 = 4$}
					\item{$10 < 15$}
					\item{$3.1415 \nq 3$}
				\end{itemize}
			\end{itemize}
		\end{minipage}
	}
\end{center}

\section{if, elif, else flow control}

The \texttt{if} statement is one of the simplest, and most powerful, opeartions that Python can perform. 
This allows the code to apply different operations \emph{if} certain criteria are \texttt{True}.
An example of a Pythonic \emph{if statement} is shown below, 
\begin{lstlisting}
# The if operator

if prior_meetings == 0:
    print("Hello World!")
\end{lstlisting}
Note that in Python the indentation is incredibly important (it is how the interpreter determines what is and is not part of the \emph{if statement}.
The above code asks the question, does the variable \texttt{prior\_meetings} has the value \texttt{0}, and if it does print the string \texttt{Hello World!}

The \texttt{if} statement may be used in a more extended context, such as if the \emph{logical argument} in the if statement is not True, an \texttt{else} can be included (or even an \texttt{elif}), as shown below,
\begin{lstlisting}
# The if operator

if prior_meetings == 0:
    print("Hello World!")
elif prior_meetings == 1:
    print("Oh! You again")
elif prior_meetings == 3:
    print("Oh! You again")
elif prior_meetings == 3:
    print("Oh! You again")
elif prior_meetings > 3:
	print("Hello old friend!")
\end{lstlisting}

\section{AND and OR operators}




In the above example, we used a logical operator with the \texttt{==} operator which translates to \emph{is equal to}. 
Other operators exist, and are shown in Table~\ref{tab:ops}.
\begin{table}[h]
	\centering
	\caption{Some logical operators available in Python.}
	\label{tab:ops}
	\begin{tabular}{c c c}
		\hline
		Name & Operator & Translation \\
		\hline
		Equals & \texttt{==} & \emph{is equal to} \\
		Less than & \texttt{<} & \emph{is less than} \\
		Less than or equal & \texttt{<=} & \emph{is less than or equal to} \\
		Greater than & \texttt{>} & \emph{is greater than} \\
		Greater than or equal to & \texttt{>=} & \emph{is greater than or equal to} \\
		Not equal & \texttt{!=} & \emph{is not equal to} \\
		\hline
	    And & \texttt{and} & \emph{this AND that} \\
	    Or & \texttt{or} & \emph{this OR that} \\
		\hline
	\end{tabular}
\end{table}
In Table~\ref{tab:ops}, the first set of operators function in the same way as the Equals operator, an example of an \emph{if statement} that improves on the code you should have written above is given below,
\begin{lstlisting}
# Complex logical comparision

if the_number == 7:
    print("I knew you'd pick 7")
elif the_number > 11:
    print("Woah! That's against the rules")
elif the_number == 10:
    print("Woah! That's against the rules")
elif the_number < 0:
    print("Woah! That's against the rules")
else:
    print("I didn't see that coming.")
\end{lstlisting}

\subsection{And and or operators}



\bibliographystyle{rsc}
\bibliography{handout_1}

\end{document}