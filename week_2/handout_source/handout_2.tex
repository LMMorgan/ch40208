\documentclass[a4paper]{article}
\usepackage{titling}
\usepackage{authblk}
\usepackage{fancyhdr}
\usepackage{hyperref}
\usepackage{rsc}
\usepackage{siunitx}
\usepackage{graphicx}
\usepackage{listings}
\usepackage{color}

\definecolor{dkgreen}{rgb}{0,0.6,0}
\definecolor{gray}{rgb}{0.5,0.5,0.5}
\definecolor{mauve}{rgb}{0.58,0,0.82}

\lstset{frame=tb,
  language=Python,
  aboveskip=3mm,
  belowskip=3mm,
  showstringspaces=false,
  columns=flexible,
  basicstyle={\ttfamily},
  numbers=none,
  numberstyle=\tiny\color{gray},
  keywordstyle=\color{blue},
  commentstyle=\color{dkgreen},
  stringstyle=\color{mauve},
  breaklines=true,
  breakatwhitespace=true,
  tabsize=3
}
\DeclareSIUnit\Fahrenheit{\degree F}

\title{Lecture 2: Pythonic logic and loops}
\author[1]{Dr Benjamin J. Morgan}
\author[1,2]{Dr Andrew R. McCluskey}
\affil[1]{Department of Chemistry, University of Bath, email: b.j.morgan@bath.ac.uk}
\affil[2]{Diamond Light Source, email: andrew.mccluskey@diamond.ac.uk}
\setcounter{Maxaffil}{0}
\renewcommand\Affilfont{\itshape\small}

\pagestyle{fancy}
\fancyhf{}
\rhead{CH40208}
\lhead{\thetitle}
\rfoot{\thepage}

\begin{document}
\maketitle

\section*{Aim}
In this lecture, you will learn about logical operations, conditional statements, and for and while loops.

\section{Logical operations}

Previously, we observed how to use Python or a Jupyter Notebook as a simple calculator. 
However, programs become more useful when we are able to make the program more ``intelligent'' allowing it to perform different calculations under different circumstances. 
The \texttt{if} statement is one of the simplest logical operation that Python can perform. 
It allows the program to apply different operations \emph{if} certain criteria are \emph{true}. 
An example of a Pythonic \emph{if statement} is shown below,
\begin{lstlisting}
# The if operator

if met_before == False:
    print("Hello World!")
\end{lstlisting}
Note that in Python the indentation is incredibly important (it is how the interpreter determines what is and is not part of the \emph{if statement}.
The above code asks the question, does the Boolean variable \texttt{met\_before} have the value \texttt{False}, and if it does print the string \texttt{Hello World!}
\vspace{\baselineskip}
\begin{center}
	\noindent\fbox{%
		\begin{minipage}{0.9\textwidth}%
			\vspace{0.15\baselineskip}
			\subsubsection{Exercise}
			\begin{itemize}
				\item{Write some code that asks the user for a number and \emph{if} the number is 7 it prints the string \texttt{I knew you'd pick 7}.}
			\end{itemize}
		\end{minipage}
	}
\end{center}
The \texttt{if} statement may be used in a more extended context, such as if the \emph{logical argument} in the if statement is not True, an \texttt{else} can be included, as shown below,
\begin{lstlisting}
# The if operator

if met_before == False:
    print("Hello World!")
else:
    print("Oh! You again")
\end{lstlisting}
In the above example, we used a logical operator with the \texttt{==} operator which translates to \emph{is equal to}. 
Other operators exist, and are shown in Table~\ref{tab:ops}.
\begin{table}[h]
	\centering
	\caption{Some logical operators available in Python.}
	\label{tab:ops}
	\begin{tabular}{c c c}
		\hline
		Name & Operator & Translation \\
		\hline
		Equals & \texttt{==} & \emph{is equal to} \\
		Less than & \texttt{<} & \emph{is less than} \\
		Less than or equal & \texttt{<=} & \emph{is less than or equal to} \\
		Greater than & \texttt{>} & \emph{is greater than} \\
		Greater than or equal to & \texttt{>=} & \emph{is greater than or equal to} \\
		Not equal & \texttt{!=} & \emph{is not equal to} \\
		\hline
	    And & \texttt{and} & \emph{this AND that} \\
	    Or & \texttt{or} & \emph{this OR that} \\
		\hline
	\end{tabular}
\end{table}


\subsection{And and or operators}


\bibliographystyle{rsc}
\bibliography{handout_1}

\end{document}